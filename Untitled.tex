% Options for packages loaded elsewhere
\PassOptionsToPackage{unicode}{hyperref}
\PassOptionsToPackage{hyphens}{url}
\documentclass[
]{article}
\usepackage{xcolor}
\usepackage[margin=1in]{geometry}
\usepackage{amsmath,amssymb}
\setcounter{secnumdepth}{-\maxdimen} % remove section numbering
\usepackage{iftex}
\ifPDFTeX
  \usepackage[T1]{fontenc}
  \usepackage[utf8]{inputenc}
  \usepackage{textcomp} % provide euro and other symbols
\else % if luatex or xetex
  \usepackage{unicode-math} % this also loads fontspec
  \defaultfontfeatures{Scale=MatchLowercase}
  \defaultfontfeatures[\rmfamily]{Ligatures=TeX,Scale=1}
\fi
\usepackage{lmodern}
\ifPDFTeX\else
  % xetex/luatex font selection
\fi
% Use upquote if available, for straight quotes in verbatim environments
\IfFileExists{upquote.sty}{\usepackage{upquote}}{}
\IfFileExists{microtype.sty}{% use microtype if available
  \usepackage[]{microtype}
  \UseMicrotypeSet[protrusion]{basicmath} % disable protrusion for tt fonts
}{}
\makeatletter
\@ifundefined{KOMAClassName}{% if non-KOMA class
  \IfFileExists{parskip.sty}{%
    \usepackage{parskip}
  }{% else
    \setlength{\parindent}{0pt}
    \setlength{\parskip}{6pt plus 2pt minus 1pt}}
}{% if KOMA class
  \KOMAoptions{parskip=half}}
\makeatother
\usepackage{graphicx}
\makeatletter
\newsavebox\pandoc@box
\newcommand*\pandocbounded[1]{% scales image to fit in text height/width
  \sbox\pandoc@box{#1}%
  \Gscale@div\@tempa{\textheight}{\dimexpr\ht\pandoc@box+\dp\pandoc@box\relax}%
  \Gscale@div\@tempb{\linewidth}{\wd\pandoc@box}%
  \ifdim\@tempb\p@<\@tempa\p@\let\@tempa\@tempb\fi% select the smaller of both
  \ifdim\@tempa\p@<\p@\scalebox{\@tempa}{\usebox\pandoc@box}%
  \else\usebox{\pandoc@box}%
  \fi%
}
% Set default figure placement to htbp
\def\fps@figure{htbp}
\makeatother
% definitions for citeproc citations
\NewDocumentCommand\citeproctext{}{}
\NewDocumentCommand\citeproc{mm}{%
  \begingroup\def\citeproctext{#2}\cite{#1}\endgroup}
\makeatletter
 % allow citations to break across lines
 \let\@cite@ofmt\@firstofone
 % avoid brackets around text for \cite:
 \def\@biblabel#1{}
 \def\@cite#1#2{{#1\if@tempswa , #2\fi}}
\makeatother
\newlength{\cslhangindent}
\setlength{\cslhangindent}{1.5em}
\newlength{\csllabelwidth}
\setlength{\csllabelwidth}{3em}
\newenvironment{CSLReferences}[2] % #1 hanging-indent, #2 entry-spacing
 {\begin{list}{}{%
  \setlength{\itemindent}{0pt}
  \setlength{\leftmargin}{0pt}
  \setlength{\parsep}{0pt}
  % turn on hanging indent if param 1 is 1
  \ifodd #1
   \setlength{\leftmargin}{\cslhangindent}
   \setlength{\itemindent}{-1\cslhangindent}
  \fi
  % set entry spacing
  \setlength{\itemsep}{#2\baselineskip}}}
 {\end{list}}
\usepackage{calc}
\newcommand{\CSLBlock}[1]{\hfill\break\parbox[t]{\linewidth}{\strut\ignorespaces#1\strut}}
\newcommand{\CSLLeftMargin}[1]{\parbox[t]{\csllabelwidth}{\strut#1\strut}}
\newcommand{\CSLRightInline}[1]{\parbox[t]{\linewidth - \csllabelwidth}{\strut#1\strut}}
\newcommand{\CSLIndent}[1]{\hspace{\cslhangindent}#1}
\setlength{\emergencystretch}{3em} % prevent overfull lines
\providecommand{\tightlist}{%
  \setlength{\itemsep}{0pt}\setlength{\parskip}{0pt}}
\usepackage{bookmark}
\IfFileExists{xurl.sty}{\usepackage{xurl}}{} % add URL line breaks if available
\urlstyle{same}
\hypersetup{
  pdftitle={Annotated Bibliography --- Bee Atlas Urbanization Project},
  pdfauthor={Andre Cachoua},
  hidelinks,
  pdfcreator={LaTeX via pandoc}}

\title{Annotated Bibliography --- Bee Atlas Urbanization Project}
\author{Andre Cachoua}
\date{}

\begin{document}
\maketitle

\section{Annotated Bibliography}\label{annotated-bibliography}

Below are the ten peer-reviewed sources used to support the proposed
research on seasonal resource gaps and bee community structure in
urbanized areas of Oregon. Each entry follows the assignment structure
and includes summary, evaluation, and relevance sections.

\begin{center}\rule{0.5\linewidth}{0.5pt}\end{center}

\subsection{Ayers \& Rehan (2021). Supporting Bees in
Cities.}\label{ayers-rehan-2021.-supporting-bees-in-cities.}

\textbf{Citation:} Ayers \& Rehan (2021)\\
\textbf{URL/DOI:} \url{https://doi.org/10.3390/insects12020128}

\textbf{Summary:}\\
This paper reviews how urban environments influence bee communities in
North America. The authors synthesize published studies on bee traits,
nesting modes, and floral resources across city gradients. They report
that cities tend to support generalist and cavity-nesting bees, while
specialist and ground-nesting species decline. The review highlights the
importance of local floral resources and landscape structure in shaping
bee diversity. It also identifies gaps such as the need to understand
seasonal effects of resource change.

\textbf{Evaluation:}\\
The paper has strong breadth and clearly organized themes, but it
depends on prior studies that vary in methods and taxonomic resolution.
Because it is a review, the conclusions rely on consistency across many
small-scale studies rather than standardized data. It remains reliable
because it cites many empirical sources and emphasizes clear mechanisms.

\textbf{Relevance:}\\
This paper supports the hypothesis that urbanization filters bee
communities toward generalists. It also helps justify measuring floral
richness and nesting traits in the Oregon Bee Atlas dataset.

\textbf{Tags:} traits, urban ecology, floral resources

\begin{center}\rule{0.5\linewidth}{0.5pt}\end{center}

\subsection{Davey et al.~(2024). Seasonal variation in urban pollen
resource
use.}\label{davey-et-al.-2024.-seasonal-variation-in-urban-pollen-resource-use.}

\textbf{Citation:} Davey et al. (2024)\\
\textbf{URL/DOI:} \url{https://doi.org/10.1007/s11252-023-01395-y}

\textbf{Summary:}\\
This study analyzed pollen collected by honey bees across urban sites in
Oslo to describe seasonal changes in floral resource use. The authors
used DNA metabarcoding to identify plant taxa across spring, summer, and
fall. They found strong seasonal turnover, where early-season pollen
came mostly from non-native ornamentals while late-season pollen came
from native species. They also observed that pollen diversity declined
late in the year. Results show that seasonal resource gaps can form even
where total plant diversity is high.

\textbf{Evaluation:}\\
The use of DNA metabarcoding provides detailed plant-level
identification, but the study centers on honey bees rather than wild bee
communities. Honey bees forage widely, which can blur fine-scale floral
shortages. Despite this limitation, the study provides high-quality
evidence of seasonal resource patterns.

\textbf{Relevance:}\\
This work supports the idea that late-season floral availability may
decline in cities. It informs the decision to test late-season bee
richness as an outcome variable.

\textbf{Tags:} phenology, pollen, floral resources

\begin{center}\rule{0.5\linewidth}{0.5pt}\end{center}

\subsection{Remmers et al.~(2024). Bees in the
City.}\label{remmers-et-al.-2024.-bees-in-the-city.}

\textbf{Citation:} Remmers \& Frantzeskaki (2024)\\
\textbf{URL/DOI:} \url{https://doi.org/10.1007/s13280-023-02064-1}

\textbf{Summary:}\\
This scoping review summarizes more than 250 studies on urban bee
ecology. It identifies consistent patterns across cities, including
higher diversity in natural areas and strong filtering toward
disturbance-tolerant bee species in urban spaces. The review highlights
how plant diversity, patch size, and green infrastructure influence bee
richness. It also notes that trait-based responses differ by region and
climate.

\textbf{Evaluation:}\\
The study is comprehensive and includes global literature, but findings
vary widely due to inconsistent sampling methods and definitions of
``urban.'' It does not provide region-specific detail for the Pacific
Northwest. However, the authors clearly outline known mechanisms and
research gaps.

\textbf{Relevance:}\\
The review strengthens the theoretical foundation for studying seasonal
resource gaps and urban filtering. It helps justify testing trait
composition and impervious surface effects in Oregon.

\textbf{Tags:} urbanization, trait filtering, review

\begin{center}\rule{0.5\linewidth}{0.5pt}\end{center}

\subsection{Rojas-Solis et al.~(2023). Arsenic and mercury tolerant
rhizobacteria.}\label{rojas-solis-et-al.-2023.-arsenic-and-mercury-tolerant-rhizobacteria.}

\textbf{Citation:} Rojas-Solis et al. (2023)\\
\textbf{URL/DOI:} \url{https://doi.org/10.1007/s00284-022-03074-5}

\textbf{Summary:}\\
This study characterizes rhizobacteria tolerant to toxic metals in
polluted soils. The authors isolated bacterial strains capable of
surviving high arsenic and mercury concentrations and evaluated their
potential roles in soil remediation. Results show that microbial
communities shift under heavy metal stress and that tolerant strains
help stabilize soils and support plant growth.

\textbf{Evaluation:}\\
Although the methods are rigorous for microbiology, the focus is not
pollinators or urban ecology. The ecological context differs from the
bee community framework. The paper is still high-quality, with clear lab
protocols and reproducible analyses.

\textbf{Relevance:}\\
This source is only indirectly relevant because it helps explain how
soil conditions can influence plant communities, which eventually
impacts floral resources. It is not central but can be cited in broader
environmental context.

\textbf{Tags:} soil microbes, tolerance, pollution

\begin{center}\rule{0.5\linewidth}{0.5pt}\end{center}

\subsection{Wilson \& Jamieson (2019). Effects of urbanization on bee
communities.}\label{wilson-jamieson-2019.-effects-of-urbanization-on-bee-communities.}

\textbf{Citation:} Wilson \& Jamieson (2019)\\
\textbf{URL/DOI:} \url{https://doi.org/10.1371/journal.pone.0225852}

\textbf{Summary:}\\
This empirical study sampled bee communities across an urban to rural
gradient in Michigan. The authors collected bees and measured plant
diversity at 15 sites. They found that urbanization increased evenness
but did not change total richness. Urban sites had more exotic and
cavity-nesting bees, and floral richness remained the strongest
predictor of diversity. Trait responses varied by nesting mode and diet
breadth.

\textbf{Evaluation:}\\
The study uses standardized sampling and includes plant and bee data,
which strengthens its conclusions. However, the gradient covers only one
region and uses moderate sample sizes. Results may not generalize to
landscapes with stronger climate gradients like Oregon.

\textbf{Relevance:}\\
The results support testing generalist vs.~specialist composition and
floral richness in the OBA dataset. It also justifies controlling for
floral host diversity in analysis.

\textbf{Tags:} gradients, traits, floral richness

\begin{center}\rule{0.5\linewidth}{0.5pt}\end{center}

\subsection{Oregon Department of Agriculture (2021). Oregon Bee Atlas
Research
Outline.}\label{oregon-department-of-agriculture-2021.-oregon-bee-atlas-research-outline.}

\textbf{Citation:} Oregon Department of Agriculture (2021)\\
\textbf{URL:} \url{https://www.oregonbeeproject.org}

\textbf{Summary:}\\
This document outlines the methods and goals of the Oregon Bee Atlas
project. It describes volunteer training, survey protocol, specimen
processing, and host-plant recording. The program uses standardized
walking surveys and detailed location records. The dataset includes
thousands of bee and plant observations across all Oregon counties.

\textbf{Evaluation:}\\
The outline is descriptive rather than analytical, but it provides
essential sampling context. The absence of raw statistical evaluation
limits its analytical depth. Still, it is authoritative because it comes
from the program managers.

\textbf{Relevance:}\\
This source explains how the OBA dataset was built, which supports
decisions on cleaning, filtering, and interpreting sampling effort.

\textbf{Tags:} methods, survey design

\begin{center}\rule{0.5\linewidth}{0.5pt}\end{center}

\subsection{Oregon Bee Atlas Survey Data 2019 (GBIF
dataset).}\label{oregon-bee-atlas-survey-data-2019-gbif-dataset.}

\textbf{Citation:} Oregon Department of Agriculture (2019)\\
\textbf{URL/DOI:} \emph{Use your GBIF download DOI}

\textbf{Summary:}\\
This dataset includes more than 25,000 bee occurrence records collected
by Oregon volunteers in 2019. It documents bee species, floral hosts,
dates, coordinates, and collection conditions. The dataset covers all
counties and includes standardized taxonomy and metadata. It also
supports cross-reference of species names and event dates.

\textbf{Evaluation:}\\
The dataset is comprehensive but has uneven sampling effort across
regions and seasons. Volunteer data require careful validation and
cleaning. Metadata fields such as coordinate uncertainty improve
transparency.

\textbf{Relevance:}\\
This dataset is the core for testing seasonal patterns, resource gaps,
and urbanization impacts in Oregon.

\textbf{Tags:} dataset, observations, Oregon

\begin{center}\rule{0.5\linewidth}{0.5pt}\end{center}

\subsection{PRISM Climate Data (2024).}\label{prism-climate-data-2024.}

\textbf{Citation:} PRISM Climate Group (2024)\\
\textbf{URL:} \url{https://prism.oregonstate.edu}

\textbf{Summary:}\\
PRISM provides monthly climate fields including temperature and
precipitation. It uses weather station data and terrain-based regression
to model climate at high resolution. The dataset is widely used for
ecological analysis. Monthly temperature is relevant for understanding
phenology and bee activity.

\textbf{Evaluation:}\\
The model uses strong interpolation methods, but coarse temporal
resolution can obscure microclimate variation. It remains one of the
most reliable climate datasets for the United States.

\textbf{Relevance:}\\
PRISM temperature helps control for climate effects when analyzing
seasonal richness in bee communities.

\textbf{Tags:} climate, temperature, seasonality

\begin{center}\rule{0.5\linewidth}{0.5pt}\end{center}

\subsection{National Land Cover Database (NLCD) Impervious Surface
(2021).}\label{national-land-cover-database-nlcd-impervious-surface-2021.}

\textbf{Citation:} U.S (2021)\\
\textbf{URL:} \url{https://www.mrlc.gov}

\textbf{Summary:}\\
NLCD provides percent impervious cover at 30 meter resolution. It uses
Landsat images and decision-tree models to classify built surfaces.
Impervious cover is a widely used index of urbanization intensity. The
dataset covers the entire United States.

\textbf{Evaluation:}\\
Accuracy varies by biome and urban density, and resolution can miss
fine-scale habitat features. Still, it is the standard dataset for
large-scale urbanization analysis.

\textbf{Relevance:}\\
This dataset will be used to measure urbanization around each OBA
sampling site to test whether late-season richness declines with
imperviousness.

\textbf{Tags:} land cover, urbanization

\begin{center}\rule{0.5\linewidth}{0.5pt}\end{center}

\section{Provenance Log}\label{provenance-log}

\begin{itemize}
\tightlist
\item
  The Oregon Bee Atlas dataset was downloaded from the Oregon Bee
  Project website as a CSV.\\
\item
  The NLCD impervious surface raster was downloaded from MRLC.\\
\item
  PRISM monthly normals were downloaded from the PRISM Climate Group.\\
\item
  All data were imported through this Rmd and processed using tidyverse,
  terra, and sf.\\
\item
  Spatial extraction of climate and impervious values was completed with
  terra::extract.\\
\item
  No data cleaning was done outside this Rmd.\\
\item
  All transformations and filtering steps are documented in code.
\end{itemize}

\phantomsection\label{refs}
\begin{CSLReferences}{1}{0}
\bibitem[\citeproctext]{ref-ayersSupportingBeesCities2021}
Ayers, A. C., \& Rehan, S. M. (2021). Supporting {Bees} in {Cities}:
{How Bees Are Influenced} by {Local} and {Landscape Features}.
\emph{Insects}, \emph{12}(2), 128.
\url{https://doi.org/10.3390/insects12020128}

\bibitem[\citeproctext]{ref-daveySeasonalVariationUrban2024}
Davey, M. L., Blaalid, R., Dahle, S., Stange, E. E., Barton, D. N., \&
Rusch, G. M. (2024). Seasonal variation in urban pollen resource use by
north temperate {European} honeybees. \emph{Urban Ecosystems},
\emph{27}(2), 515--529. \url{https://doi.org/10.1007/s11252-023-01458-1}

\bibitem[\citeproctext]{ref-oregonBeeAtlasSurvey2019}
Oregon Department of Agriculture. (2019). \emph{Oregon {Bee Atlas Survey
Data} 2019}. \url{https://doi.org/10.5399/osu/cat_osac.6.1.4906}

\bibitem[\citeproctext]{ref-oregondepartmentofagricultureOregonBeeAtlas2021}
Oregon Department of Agriculture. (2021). \emph{Oregon {Bee Atlas
Research Outline} (2018--2021)}. Oregon Department of Agriculture.

\bibitem[\citeproctext]{ref-prismclimategroupPRISMClimateData2024}
PRISM Climate Group, : (2024). \emph{{PRISM Climate Data} (30-arc-second
resolution)}.

\bibitem[\citeproctext]{ref-remmersBeesCityFindings2024}
Remmers, R., \& Frantzeskaki, N. (2024). Bees in the city: {Findings}
from a scoping review and recommendations for urban planning.
\emph{Ambio}, \emph{53}(9), 1281--1295.
\url{https://doi.org/10.1007/s13280-024-02028-1}

\bibitem[\citeproctext]{ref-rojas-solisArsenicMercuryTolerant2023}
Rojas-Solis, D., Larsen, J., \& Lindig-Cisneros, R. (2023). Arsenic and
mercury tolerant rhizobacteria that can improve phytoremediation of
heavy metal contaminated soils. \emph{PeerJ}, \emph{11}, e14697.
\url{https://doi.org/10.7717/peerj.14697}

\bibitem[\citeproctext]{ref-nlcd2021}
U.S, G. S. (2021). \emph{National {Land Cover Database} ({NLCD}) 2021
{Impervious Surface Product}}.

\bibitem[\citeproctext]{ref-wilsonEffectsUrbanizationBee2019}
Wilson, C. J., \& Jamieson, M. A. (2019). The effects of urbanization on
bee communities depends on floral resource availability and bee
functional traits. \emph{PLOS ONE}, \emph{14}(12), e0225852.
\url{https://doi.org/10.1371/journal.pone.0225852}

\end{CSLReferences}

\end{document}
